\newpage
\begin{center}
\large \bf
PREDYKCJA POZIOMU ZAPYLENIA POWIETRZA W WARSZAWIE
\end{center}

\section*{Streszczenie}
W pracy opisany został problem zapylenia smogowego w obszarach zurbanizowanych oraz został zaproponowany sposób rozwiązania go używając powiązań między zmiennymi pogodowymi i smogowymi. Zrobiony został przegląd literatury dotyczącej dotychczasowych sposobów  rozwiązania tego problemu. Zebrane zostały dane pogodowe i ze wskaźników zapylenia. Przeprowadzona została eksploracyjna analiza danych, w celu wydobycia wiedzy na temat powiązań między pogodą, a zapyleniem. Zbudowany został model wyjaśniający te powiązania. Następnie ten model został przetestowany i przedstawione zostały wyniki. W podsumowaniu autor przedstawił możliwe dalsze kierunki rozwoju pracy.

\bigskip
{\noindent\bf Słowa kluczowe:} Eksploracyjna analiza danych, Uczenie maszynowe, Modelowanie statystyczne, Smog, Zapylenie powietrza

\vskip 2cm


\begin{center}
\large \bf
PREDICTION OF SMOG AIR POLLUTION IN WARSAW
\end{center}

\section*{Abstract}
In the thesis, the problem of smog air pollution in urban regions has been described. The author proposed the possible approach to solving it using the relationships between the weather and pollution data. Author researched the solutions that have been published up to this point. Author gathered the data and went through the process of data mining in order to extract the knowledge. The statistical model have been created that explains the relationships in the data. The model has been tested and the effects have been presented. In the summary, the author have outlined the possible next steps in regard to this work.

\bigskip
{\noindent\bf Keywords:} Data mining, Exploratory data analysis, Machine learning, Statistical modelling, Air pollution

\vfill